\chapter{Kinematics}
\label{chap:kinematics}

The 5-bar linkage design of the leg was first designed constructed in the study \cite{Duperret}. The geometry of the leg is fairly complex and the derivation of the kinematic equations equally so. J.M. Duperret and D.E. Koditschek derived the kinematic equations \cref{eq:forward-kinematics, eq:reverse-kinematics} in the study \cite{Duperret}. 

In this study the assumption was made that the distance $d$, as seen in \cref{fig:Geometric view of leg}, is zero. This simplifies the derivation of forward and reverse kinematic equations of the leg design by making the leg a 4-bar linkage. These kinematic equations are more easily calculated on board a microcontroller\cite{Duperret}, leaving more processing power for other control tasks if needed. 

The ease of calculation makes the loss in accuracy acceptable - in practise the simplified kinematic equations worked well with an insignificant calculation time made possible by the STM32F4's on-board floating point unit.

\begin{equation} \label{eq:forward-kinematics}
f(\phi_1, \phi_2) = \left(\begin{array}{c} \sqrt{{\mathrm{l_2}}^2 - {\mathrm{l_1}}^2\, {\sin\!\left(\frac{\mathrm{\phi_1}}{2} + \frac{\mathrm{\phi_2}}{2}\right)}^2} - \mathrm{l_1}\, \cos\!\left(\frac{\mathrm{\phi_1}}{2} + \frac{\mathrm{\phi_2}}{2}\right)\\
\frac{\mathrm{\phi_1}}{2} - \frac{\mathrm{\phi_2}}{2} \end{array}\right)
\end{equation}

\begin{equation} \label{eq:reverse-kinematics}
g(r, \theta) = \left(\begin{array}{c} \pi - acos(\frac{r^2 + l_1^2 - l_2^2}{2rl_1}) + \theta \\
\pi - acos(\frac{r^2 + l_1^2 - l_2^2}{2rl_1}) - \theta  \end{array}\right)
\end{equation}

\section{The Jacobian}
The Jacobian is formed by taking partial derivatives of the forward kinematic equation \cref{eq:forward-kinematics} as shown in \cref{eq:jacobian}. 

It is used as a mapping from the joint angles $\phi_1$ and $\phi_1$ to the end effector generalized coordinates $r$ and $\theta$. The Jacobian can be applied in robotic kinematic control to determine joint velocities and forces to achieve a desired force or velocity at the end effector, in this case the leg foot.

Taking the Jacobian of the kinematic mapping $f(\phi_1, \phi_2)$ the foot force vector, F, can be transformed to the motor torque commands, $\tau$:
\begin{equation} \label{eq:jacobian}
J = \left[ \frac{\partial \textbf{f}}{\partial \textbf{X}} \right] 
\end{equation}
where \textbf{X} = [r $\theta$].

\subsection{Velocity Mapping}

Velocity mapping from motor rotational velocity to polar velocity.
\begin{equation}
v(\dot{r}, \dot{\theta}) = J w(\dot{\phi_1}, \dot{\phi_2})
\end{equation}

