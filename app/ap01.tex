\chapter{Communication Protocol Code}
\label{app:Communication Protocol Code}
Included pieces of code that may not be obvious to users or code that was particularly important to the operation of the protocol.

%\inputminted{octave}{BitXorMatrix.m}

\begin{listing}[ht]
\begin{minted}[
linenos,
bgcolor=smokyblack]{c}
//Communication Timing
uint8_t Ts = 25; //Sampling time in 1/_X_ ms
uint8_t Td = 3;
//NB: #define configTICK_RATE_HZ ((TickType_t)_X_000) in FreeRTOSConfig.h
\end{minted}
\caption{FreeRTOS timing configuration.}
\label{listing:FreeRTOS timing}
\end{listing}

\begin{listing}[ht]
\begin{minted}[
linenos,
bgcolor=smokyblack]{c}
struct __attribute__((__packed__)) RXPacketStruct {
	uint8_t START[2];
	...
	uint8_t StatBIT_1 : 1; //Bit field
	uint8_t StatBIT_2 : 1;
	uint8_t StatBIT_3 : 1;
	...
	uint8_t CRCCheck[2]; //CRC-CCITT        
	uint8_t STOP[2];
};
\end{minted}
\caption{PC RX "packed" packet structure.}
\label{listing:packed-packet}
\end{listing}

\begin{listing}[ht]
\begin{minted}[
linenos,
bgcolor=smokyblack]{c}
union {
	uint32_t WORD;
	uint16_t HALFWORD;
	uint8_t BYTE[4];
} WORDtoBYTE;
\end{minted}
\caption{Byte conversion union.}
\label{listing:Byte conversion union}
\end{listing}

\begin{listing}[ht]
\begin{minted}[
linenos,
bgcolor=smokyblack]{c}
HAL_UART_Receive_DMA(&PC_UART, RXBufPC, sizeof(RXPacket));
if(xSemaphoreTake( PCRXHandle, portMAX_DELAY ) == pdTRUE) {
	rcvdCount = sizeof(RXPacket);
	START_INDEX = findBytes(RXBufPC, rcvdCount, 
	RXPacket.START, 2, 1);
	if(START_INDEX>=0) {
		memcpy(RXPacketPTR, &RXBufPC[START_INDEX], 		
		sizeof(RXPacket));
		RX_DATA_VALID = 0;

		WORDtoBYTE.BYTE[1] = RXPacket.CRCCheck[0];
		WORDtoBYTE.BYTE[0] = RXPacket.CRCCheck[1];
		CALC_CRC = crcCalc(&RXPacket.OPCODE, 0, PAYLOAD_RX, 0);
		
		//A useful tool when calculating and 
		//confirming CRC values of various types: 
		//https://www.lammertbies.nl/comm/info/crc-		
		//calculation.html

		if(WORDtoBYTE.HALFWORD==CALC_CRC) {
			RX_DATA_VALID = 1;
			... //Packet processing
\end{minted}
\caption{PC RX packet processing.}
\label{listing:PC RX packet processing}
\end{listing}

\begin{listing}[ht]
\begin{minted}[
linenos,
bgcolor=smokyblack]{c}
struct __attribute__((__packed__)) BaseCommandStruct {
        uint8_t START[2];
        uint8_t CB;
        uint8_t INDOFF[2];
        uint8_t LEN;
        uint8_t CRC1[2];
        uint8_t DATA[4];
        uint8_t CRC2[2];
};

uint8_t SNIP;

struct BaseCommandStruct BaseCommand[50];
struct BaseCommandStruct* BaseCommandPTR;
\end{minted}
\caption{Motor packet array of structs.}
\label{listing:Motor packet array of structs}
\end{listing}

\begin{listing}[ht]
\begin{minted}[
linenos,
bgcolor=smokyblack]{c}
BaseCommandPTR = &BaseCommand[RXPacket.OPCODE];
BaseCommandCompile(RXPacket.OPCODE, 0b0011, 0x02, 0x45, 0x02,
RXPacket.M1C, 2, 0);
xQueueOverwrite(ICommandM1QHandle, &BaseCommandPTR);
\end{minted}
\caption{Motor packet compilation current command example.}
\label{listing:Motor packet compilation example}
\end{listing}